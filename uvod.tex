\chapter{Uvod}
\label{chUvo}

Zelo pomemben je jezik ter razumljivost pisanja. Besedilo mora biti pripravljeno 
v skladu s pravili za objavo v znanstveni reviji.  Nekaj koristnih nasvetov 
o strokovnem pisanju najdete v "clanku prof. Ivana Ku"s"cerja : O strokovnem pisanju 
\cite{Ku}.  Pomagate si lahko tudi z navodili Ameri"skega fizikalnega dru"stva 
\cite{APS}.\\

Vse strani morajo biti "stete, tudi prazne strani, priloge med besedilom in na koncu 
dela.  Osrednje besedilo mora biti o"stevil"ceno z arabskimi "stevilkami, za"cetne 
in kon"cne strani pa so lahko o"stevil"cene tudi z rimskimi "stevilkami.  "Stevilke 
morajo biti izpisane na spodnjem delu strani. Tisk naj bo razen za"cetnih strani 
dvostranski.  Obvezna je vezava v trde platnice.\\

{\bf Vrstni red vsebine:}

\begin{itemize}[noitemsep]
\item{Naslovna stran}
\item{Zahvala (neobvezno)}
\item{Izvle"cek v slovenskem jeziku. Dodajte tudi klju"cne besede v slovenskem jeziku}
\item{Izvle"cek v angle"skem jeziku. Dodajte tudi klju"cne besede v angle"skem jeziku}
\item{Kazalo vsebine}
\item{Uvod}
\item{Osrednji del}
\item{Zaklju"cni del}
\item{Seznam literature}
\item{Dodatki (neobvezno)}
\item{Stvarno kazalo (neobvezno)}
\end{itemize}

Literatura mora biti citirana "ze v besedilu. Citirani viri sproti povedo, kje naj 
bralec i"s"ce dodatne informacije.  Seznam naj bo urejen po vrstnem redu, kot se 
navedbe pojavijo v delu.  V primeru uporabe programa {Bib\TeX} za navajanje literature 
izberite Bib\TeX{ov} stil, prirejen po apsrev4-2.bst, ki ga najdete na spletni strani 
poleg te predloge.  Navodila za delo s programom {Bib\TeX} najdete na spletni strani 
\cite{Bib}, navodila za namestitev paketa REVTeX 4.1 pa na strani \cite{Rev}.  Za vnos 
bibliografskih enot priporo"camo uporabo programa \href{http://www.jabref.org/}{JabRef} 
\cite{JR}.  V seznamu literature te predloge smo naslove spletnih strani in online 
dokumentov vnesli v polje {\tt Note} pred datumom ogleda spletne strani.  "Ce "zelite, 
da se vam v seznamu literature elektronski naslov ne izpi"se, ga vnesite v polje 
{\tt URL}.  V vseh primerih, kjer je to mo"zno, dodajte aktivne povezave za dostop 
do elektronskih dokumentov.  Pred izpolnjevanjem polj obvezno preberite navodila 
v pomo"ci (User Documentation). Tu je izsek iz navodil: 
\href{https://docs.jabref.org/advanced/fields}{About BibTeX and its fields} \cite{Help}.\\

\index{Bib\TeX}
\begin{filecontents*}[overwrite]{\jobname.xmpdata}
\Title{Naslov magistrskega dela} 
\Author{Tina Klobas}
\Keywords{ključna beseda 1\sep ključna beseda 2\sep ključna beseda 3}
\Subject{Fizika}
\end{filecontents*}

%-------------------------------------------------------------------------------------------------

\documentclass[longbibliography,slovene,a4paper,12pt]{book}

%-----------------------------------------------------------------------------------
%     PDF/A
%----------------------------------------------------------------------------------

\usepackage{xmpincl}
\usepackage[a-1b, mathxmp]{pdfx}[2018/12/22]

%-------------------------------------------------------------------------------------

\usepackage[slovene]{babel}    
\usepackage[utf8]{inputenc}
\usepackage{amsfonts, amsmath}                      %% amsmath for \text{} to work
\usepackage{listings}                               %% for pseudocode
\usepackage[T1]{fontenc}
\usepackage[pdftex]{graphicx}
\usepackage{xcolor,colortbl}
\usepackage{fancyhdr}
\usepackage[explicit,compact]{titlesec}
\titleformat{\chapter}[block]
{\bfseries\huge}{\filright\huge\thechapter.}{1ex}{\huge\filright #1}
\usepackage[sort,numbers]{natbib}
\usepackage[nottoc,numbib]{tocbibind}
\usepackage{acro}
\usepackage{filecontents}
\usepackage{hyperref}
\usepackage{url}
\usepackage[a4paper,inner=3.5cm,outer=2.5cm,top=2.5cm,bottom=2.5cm,pdftex]{geometry}
\usepackage[titletoc, title]{appendix}
\usepackage[outdir=./gnuplot/]{epstopdf}               %% set directory of tex files
\usepackage{makeidx}
\setlength{\headheight}{15pt}
\usepackage{enumitem}
\usepackage{emptypage}
\usepackage{tocloft}
\renewcommand{\cftpartleader}{\cftdotfill{\cftdotsep}} 
\renewcommand{\cftchapleader}{\cftdotfill{\cftdotsep}} 
\usepackage{caption}

\usepackage{standalone}                           %% needed for external tikz files
\usepackage{figures/TikzPack}                     %% needed for external tikz files

\def\epsfg#1#2{\epsfig{file=#1.eps,width=#2}}
\def\legendamp#1#2{\vbox{\hsize=#1\caption{\small #2}}}

\setcounter{topnumber}{4}
\setcounter{bottomnumber}{4}
\setcounter{totalnumber}{5}
\renewcommand{\topfraction}{0.99}
\renewcommand{\bottomfraction}{0.99}
\renewcommand{\textfraction}{0.0}
\setlength{\tabcolsep}{10pt}
\renewcommand{\arraystretch}{1.5}

\def\bi#1{\hbox{\boldmath{$#1$}}}
\let\oldvec\vec
\def\vec#1{\mbox{\boldmath$#1$}}
\def\pol{{\textstyle{1\over2}}}
\def\svec#1{\mbox{{\scriptsize \boldmath$#1$}}}

\makeindex
%-----------------------------------------------------------------------------------
%    SUMNIKI
%-----------------------------------------------------------------------------------
%  Za pisanje sumnikov imamo tri moznosti:
%   --- vnasamo jih neposredno v kodnem sistemu UTF-8 (priporocljivo)
%   --- pisemo jih z latexovim ukazom, ki je namenjen natanko temu,
%       in sicer kot \v{c}, \v{s}, \v{z}, \v{C}, \v{S}, v{\Z} ali
%       malo manj pregledno kot \v c, \v s, \v z, \v C, \v S, \v Z,
%   --- pisemo jih kot "c, "s, "z, "C, "S, "Z, vendar tedaj potrebujemo
%       spodaj zapisani macro, ki znaku " pripise vlogo `izdelave' sumnika:
\catcode`\"=\active\def"#1{\v{#1}}
%       torej \v{S}krjan\v{c}ek == \v Skrjan\v cek == "Skrjan"cek
%  Pozor: narekovaj potem ne smemo vec pisati kot " ampak kot `` in ''
%       torej: "Skrjan"cek je "civkal ``"ci-"ci-"ci''.

\newcommand{\e}{\mathrm{e}}

%-----------------------------------------------------------------------------------
%   PRIPOROCILO
%
%  V primeru, da je besedilna datoteka prevelika za pregledno urejanje,
%  priporocamo, da vsebino razdelite na posamezna poglavja, ki jih v glavni
%  dokument vkljucite z ukazom \input{naslov_poglavja}.
%
%-----------------------------------------------------------------------------------

%% dodam svoje barve, da bojo povsod enake
% \definecolor{myblue}{HTML}{043565}
% \definecolor{mymagenta}{HTML}{8f2d56}
% \definecolor{myteal}{HTML}{218380}
% \definecolor{mytangerine}{HTML}{fb9f89}
% \definecolor{mygray}{HTML}{9a9aac}
% \definecolor{myyellow}{HTML}{ffc43d}

\definecolor{myred}{HTML}{f94144}
\definecolor{myredorange}{HTML}{f3722c}
\definecolor{myorange}{HTML}{f8961e}
\definecolor{myyellow}{HTML}{f9c74f}
\definecolor{mylightgreen}{HTML}{90be6d}
\definecolor{mygreen}{HTML}{43aa8b}
\definecolor{myblue}{HTML}{577590}

\begin{document}

%-----------------------------------------------------------------------------------
%       NASLOVNA STRAN 
%-----------------------------------------------------------------------------------

\pagestyle{empty}
\begin{center}

{\large UNIVERZA V LJUBLJANI\\
 FAKULTETA ZA MATEMATIKO IN FIZIKO\\
ODDELEK ZA FIZIKO\\
FIZIKA II.~STOPNJA, FIZIKA KONDENZIRANE SNOVI\\}


\vspace{4cm}
{\Large Tina Klobas\\}
\vspace{10mm}
{\bf \Large ORIENTACIJSKO ZAGOZDENJE DVODIMENZIOALNIH ELIPS V RAVNINI}\\
\vspace{5mm}
{\large Magistrsko delo}\\

\vfill

{\large MENTOR: dr. Anže Božič\\
SOMENTOR: doc. dr. Simon Čopar\\

\vspace{2cm}
Ljubljana, 2023}

\end{center}

%-----------------------------------------------------------------------------------
%        ZAHVALA (NEOBVEZNO)
%-----------------------------------------------------------------------------------

% \cleardoublepage
% \mbox{}
% \vfill
% {\Large \bf Zahvale}
% \vspace{1cm}\\
% Na tem mestu zapi"site, komu se zahvaljujete za pomo"c pri nastanku magistrskega 
% dela.

%-----------------------------------------------------------------------------------
%         IZVLECEK
%-----------------------------------------------------------------------------------

% \cleardoublepage
\pagestyle{plain}
\begin{center}
{\bf Naslov v slovenskem jeziku}\\[3mm]
{\sc  Izvle"cek}
\end{center}
\vspace{10mm}
Kratek izvle"cek v slovenskem jeziku, do 300 besed.\\[10mm]
{\bf Klju"cne besede:}\\[3mm]


\cleardoublepage
\foreignlanguage{english}{  %  angleski delilni vzorci
\begin{center}
{\bf Naslov v angleškem jeziku}\\[3mm]
{\sc  Abstract}
\end{center}
\vspace{10mm}
Kratek izvleček v angleškem jeziku, do 300 besed.\\[10mm]
{\bf Keywords:}\\[3mm]
}

%-----------------------------------------------------------------------------------
%    KAZALO
%-----------------------------------------------------------------------------------

% \cleardoublepage
% \tableofcontents

%-----------------------------------------------------------------------------------
%       OSREDNJI DEL
%-----------------------------------------------------------------------------------

\cleardoublepage
\pagestyle{headings}
\fancyhead[CE,RE]{}
\fancyhead[LO,CO]{}
\fancyhead[LE]{\textbf{\nouppercase{\leftmark}}}
\fancyhead[RO]{\textbf{\nouppercase{\rightmark}}}


%\chapter{Uvod}
\label{chUvo}

Zelo pomemben je jezik ter razumljivost pisanja. Besedilo mora biti pripravljeno 
v skladu s pravili za objavo v znanstveni reviji.  Nekaj koristnih nasvetov 
o strokovnem pisanju najdete v "clanku prof. Ivana Ku"s"cerja : O strokovnem pisanju 
\cite{Ku}.  Pomagate si lahko tudi z navodili Ameri"skega fizikalnega dru"stva 
\cite{APS}.\\

Vse strani morajo biti "stete, tudi prazne strani, priloge med besedilom in na koncu 
dela.  Osrednje besedilo mora biti o"stevil"ceno z arabskimi "stevilkami, za"cetne 
in kon"cne strani pa so lahko o"stevil"cene tudi z rimskimi "stevilkami.  "Stevilke 
morajo biti izpisane na spodnjem delu strani. Tisk naj bo razen za"cetnih strani 
dvostranski.  Obvezna je vezava v trde platnice.\\

{\bf Vrstni red vsebine:}

\begin{itemize}[noitemsep]
\item{Naslovna stran}
\item{Zahvala (neobvezno)}
\item{Izvle"cek v slovenskem jeziku. Dodajte tudi klju"cne besede v slovenskem jeziku}
\item{Izvle"cek v angle"skem jeziku. Dodajte tudi klju"cne besede v angle"skem jeziku}
\item{Kazalo vsebine}
\item{Uvod}
\item{Osrednji del}
\item{Zaklju"cni del}
\item{Seznam literature}
\item{Dodatki (neobvezno)}
\item{Stvarno kazalo (neobvezno)}
\end{itemize}

Literatura mora biti citirana "ze v besedilu. Citirani viri sproti povedo, kje naj 
bralec i"s"ce dodatne informacije.  Seznam naj bo urejen po vrstnem redu, kot se 
navedbe pojavijo v delu.  V primeru uporabe programa {Bib\TeX} za navajanje literature 
izberite Bib\TeX{ov} stil, prirejen po apsrev4-2.bst, ki ga najdete na spletni strani 
poleg te predloge.  Navodila za delo s programom {Bib\TeX} najdete na spletni strani 
\cite{Bib}, navodila za namestitev paketa REVTeX 4.1 pa na strani \cite{Rev}.  Za vnos 
bibliografskih enot priporo"camo uporabo programa \href{http://www.jabref.org/}{JabRef} 
\cite{JR}.  V seznamu literature te predloge smo naslove spletnih strani in online 
dokumentov vnesli v polje {\tt Note} pred datumom ogleda spletne strani.  "Ce "zelite, 
da se vam v seznamu literature elektronski naslov ne izpi"se, ga vnesite v polje 
{\tt URL}.  V vseh primerih, kjer je to mo"zno, dodajte aktivne povezave za dostop 
do elektronskih dokumentov.  Pred izpolnjevanjem polj obvezno preberite navodila 
v pomo"ci (User Documentation). Tu je izsek iz navodil: 
\href{https://docs.jabref.org/advanced/fields}{About BibTeX and its fields} \cite{Help}.\\

\index{Bib\TeX}
\lstset{language=Python}
\lstset{frame=tb,
  language=Python,
  aboveskip=3mm,
  belowskip=3mm,
  showstringspaces=false,
  columns=flexible,
  basicstyle={\small\ttfamily},
  numbers=none,
  numberstyle=\tiny\color{gray},
  keywordstyle=\color{myteal},
  commentstyle=\color{myblue},
  stringstyle=\color{myyellow},
  breaklines=true,
  breakatwhitespace=true,
  tabsize=3
}

\chapter{Metodologija}
\label{chMe}

Okviren potek zagozdenja sistema:
\begin{enumerate}
\item{Generiranje dvdimenzionalne mreže $N$ točk z~Mitchellovim algoritmom~\cite{mitch}}.
\item{Postavitev $N$ elips z~ekscentričnostjo $e$ in naključnimi začetnimi orientacijami.}
\item{Implementacija prekrivalne funkcije~\cite{perram1985} za zaznavanje trkov med elipsami.}
\item{Postopno večanje elips in relaksacija vrtenja (Monte Carlo).}
\item{Analiza konfiguracij.}
\end{enumerate}
Dvodimenzionalnemu prostoru dodamo še periodične robne pogoje, ki jih vključimo na naslednji 
način:
\begin{lstlisting}[frame=single]  
    dx = point2 - point1
    if dx > 0.5*width:
        dx = dx - width
    elif dx < -0.5*width:
        dx = dx + width
\end{lstlisting}

\section{Mitchellov algoritem}

\begin{figure}[h!]
    \centering
    \input{./gnuplot/porazdelitev_1024.tex}
    \caption[Blue noise]{Porazdelitev $1024$ točk.}
    \label{slika1}
\end{figure}

Okviren potek algoritma:
\begin{enumerate}
    \item{za začetni točki zgeneriramo naključni poziciji,}
    \item{zgeneriramo naključne pozicije kandidatov za naslednjo točko,}
    \item{izberemo tistega kandidata, ki je najdlje od vseh točk porazdelitve 
    (ima največjo minimalno oddaljenost),}
    \item{ponovimo korak 2 dokler ne dobimo željeno število točk.}
\end{enumerate}
Med algoritmom število kandidatov povečujemo sorazmerno s~številom že obstoječih
točk $N$. Pri nalogi smo tako na vsakem koraku generirali 
$ \left \lfloor{N/2}\right \rfloor + 1$ kandidatov.\\
%% napisi se nek del kako je to kul in primerjavo z modrim sumom blabla

\section{Eliptična kontaktna funkcija}

Kontaktna funkcija je dober kriterij za prekrivanje med dvema elipsama.\footnote{V~nalogi
uporabljamo dvodimenzionalni ekvivalent tridimenzionalne funkcije povzete 
po~\cite{perram1985}.}

\subsection{Ena elipsa}
Elipso $A$ definiramo s~funkcijo $E_a$, ki zadošča naslednjemu pogoju
\begin{equation}
    E_a (\vec{r}-\vec{r}_a, \theta_a) 
    \begin{cases}
        \quad <1 & \text{znotraj A}\\
        \quad =1 & \text{na površini A}\\
        \quad >1 & \text{zunaj A}
      \end{cases}. 
    \label{eq:cases}
\end{equation}
Izberemo

\begin{equation}
    E_a = (\vec{r} - \vec{r}_a)^{\top} \mathbf{A}^{-1} (\vec{r} - \vec{r}_a),
\label{eq:elipsa}
\end{equation}
kjer je $\vec{r}_a$ center elipse. Matriko $\mathbf{A}$ lahko zapišemo kot
\begin{equation}
    \mathbf{A} (\theta_a) = \mathbf{R}(\theta_a) \hat{\mathbf{A}} 
                            \mathbf{R}^{\top}(\theta_a),
\end{equation}
kjer je $R(\theta)$ rotacijska matrika, $\hat{\mathbf{A}}$ pa je definirana z~velikostjo 
polosi elipse $a_1$ in $a_2$ ter enotskih vektorjev $\vec{e}_i$ 
\begin{equation}
    \hat{\mathbf{A}} = \sum_{i=1}^{2} a_i^2 \vec{e}_i \vec{e}_i^{\top}.
\end{equation}
Matriki $\mathbf{A}$ in $\mathbf{A}^{-1}$ sta simetrični in pozitivno definitni.

\subsection{Prekrivanje dveh elips}
Definiramo funkcijo
\begin{equation}
    F(\vec{r}, \lambda) = \lambda E_a (\vec{r}) + (1 - \lambda) E_a (\vec{r}),
    \label{eq:f}
\end{equation}
ki je odvisna od pozicij $\vec{r}_a, \, \vec{r}_b$ in orientacij $\theta_a, \, \theta_b$
dveh elips $A$ ter $B$. Parameter $\lambda$ omejimo na interval $[0,1]$, tako, da je 
$F(\vec{r}, \lambda) \geq 0$. Pri fiksni vrednosti $\lambda$ ima $F(\vec{r}, \lambda)$
enolični minimum. Pri $\lambda=0$ je minimum $F=0$ pri $\vec{r} = \vec{r}_b$, pri 
$\lambda=1$ pa je minimum $F=0$ pri $\vec{r} = \vec{r}_a$. Za vse vmesne vrednosti
$\lambda$, je vrednost $\vec{r}$ pri minimumu $F(\vec{r}, \lambda)$ določena z
\begin{equation}
    \nabla F(\vec{r}, \lambda) = 0,
\end{equation}
oziroma
\begin{equation}
    \lambda \mathbf{A}^{-1} (\vec{r} - \vec{r}_a) + (1-\lambda) \mathbf{B}^{-1}
    (\vec{r} - \vec{r}_b) = 0.
\end{equation}
To lahko napišemo tudi kot
\begin{align}
    \vec{r}(\lambda) &- \vec{r}_a = (1-\lambda) \mathbf{A} \mathbf{C}^{-1} 
        \vec{r}_{a\,b}, \nonumber \\
    \vec{r}(\lambda) &- \vec{r}_b = - \lambda \mathbf{B} \mathbf{C}^{-1} \vec{r}_{a\,b}, 
    \label{eq:rab}
\end{align}
kjer je $\vec{r}_{a\,b} = \vec{r}_b - \vec{r}_a$ in $\mathbf{C}$ matrika
\begin{equation}
    \mathbf{C} = (1-\lambda) \mathbf{A} + \lambda \mathbf{B}.
    \label{eq:C}
\end{equation}
Rešitev~\ref{eq:rab} ustavimo v~\ref{eq:f}, pri čemer upoštevamo tudi~\ref{eq:C}
in definiramo funkcijo $f$ kot
\begin{equation}
    f(\lambda) = F (\vec{r}(\lambda), \lambda) = \lambda (1-\lambda)
        \vec{r}_{a\,b}^{\top} C^{-1} \vec{r}_{a\,b}.
\end{equation}


%\input{zakljucek}

%------------------------------------------------------------------------
%       LITERATURA
%------------------------------------------------------------------------

\cleardoublepage
\renewcommand\bibname{Literatura}
\bibliographystyle{apsrev4-2-fmf-slo}
\bibliography{Bibliografija-slo}


%-----------------------------------------------------------------------
%       DODATKI
%-----------------------------------------------------------------------

\cleardoublepage\phantomsection
\titleformat{\chapter}[display]
{\bfseries\huge}{\chaptertitlename\ \thechapter}{1ex}{\huge\filright #1}


% \begin{appendices}   

%\input{Naslov prvega dodatka}
%\input{Naslov drugega dodatk}

% \end{appendices}


%-----------------------------------------------------------------------
%       KAZALO (NEOBVEZNO)
%-----------------------------------------------------------------------

% \cleardoublepage
% \printindex

\end{document}